\section{Lecture 3}

\begin{definition}
    Let $\bb{X}$ be a normed vector space.

    \begin{itemize}
        \item $\Gamma(\bb{X}) := \{ f : \bb{X} \to \bb{R} \mid f \text{ is convex} \}$.

        \item $C^0(\bb{X})$ denotes the space of continuous functions on $\bb{X}$.

        \item $C^1(\bb{X})$ denotes the space of continuously differentiable functions on $\bb{X}$.

        \item A function $f : \bb{X} \to \bb{R}$ is called $\beta$-Lipschitz if
              \[
                  |f(\bd{x}) - f(\bd{y})|
                  \le \beta \|\bd{x} - \bd{y}\|
                  \quad \text{for all } \bd{x},\bd{y} \in \bb{X}.
              \]

        \item A function $f : \bb{X} \to \bb{R}$ is called $\beta$-smooth if
              $f \in C^1(\bb{X})$ and
              \[
                  \|\nabla f(\bd{x}) - \nabla f(\bd{y})\|
                  \le \beta \|\bd{x} - \bd{y}\|
                  \quad \text{for all } \bd{x},\bd{y} \in \bb{X}.
              \]

        \item
              \[
                  C^1_\beta(\bb{X})
                  := \{ f : \bb{X} \to \bb{R}
                  \mid f \in C^1(\bb{X}) \text{ and }
                  \nabla f \text{ is } \beta\text{-Lipschitz} \}.
              \]
    \end{itemize}
\end{definition}

\begin{examples}
    \begin{itemize}
        \item Let $\bb{X}=\bb{R}^2$. The function
              \[
                  f(\bd{x}) = \tfrac12 \|\bd{x}\|_2^2
              \]
              is $1$-smooth.

        \item Let $Q \in \bb{R}^{n \times n}$ be symmetric positive semidefinite. The quadratic function
              \[
                  f(\bd{x}) = \tfrac12 \bd{x}^\top Q \bd{x}
              \]
              is $\lambda_{\max}(Q)$-smooth.

        \item The function
              \[
                  f(\bd{x}) = \|\bd{x}\|_2^2
              \]
              is $4$-smooth on the unit ball
              \[
                  \{ \bd{x} \in \bb{R}^n : \|\bd{x}\|_2 \le 1 \}.
              \]
    \end{itemize}
\end{examples}


\begin{theorem}
    Let $f : \bb{R}^n \to \bb{R}$ be continuously differentiable.
    Then $f$ is $\beta$-Lipschitz (with respect to the Euclidean norm) if and only if
    \[
        \|\nabla f(\bd{x})\|_2 \le \beta
        \quad \text{for all } \bd{x} \in \bb{R}^n.
    \]
\end{theorem}

\begin{proof}
    ($\Rightarrow$)
    Assume $f$ is $\beta$-Lipschitz.
    Fix $\bd{x} \in \bb{R}^n$ and let $\bd{v} \in \bb{R}^n$ be any unit vector.
    For $t \in \bb{R}$,
    \[
        |f(\bd{x} + t\bd{v}) - f(\bd{x})|
        \le \beta |t|.
    \]
    Dividing by $|t|$ and letting $t \to 0$, we obtain
    \[
        |\nabla f(\bd{x})^\top \bd{v}| \le \beta.
    \]
    Taking the supremum over all unit vectors $\bd{v}$ yields
    \[
        \|\nabla f(\bd{x})\|_2 \le \beta.
    \]

    ($\Leftarrow$)
    Assume $\|\nabla f(\bd{x})\|_2 \le \beta$ for all $\bd{x} \in \bb{R}^n$.
    For any $\bd{x}, \bd{y} \in \bb{R}^n$, define
    \[
        \bld{\gamma}(t) = \bd{x} + t(\bd{y} - \bd{x}),
        \quad t \in [0,1].
    \]
    By the fundamental theorem of calculus,
    \[
        f(\bd{y}) - f(\bd{x})
        = \int_0^1 \nabla f(\bld{\gamma}(t))^\top (\bd{y} - \bd{x}) \, dt.
    \]
    Applying the Cauchy--Schwarz inequality,
    \[
        |f(\bd{y}) - f(\bd{x})|
        \le \int_0^1 \|\nabla f(\bld{\gamma}(t))\|_2 \, \|\bd{y} - \bd{x}\|_2 \, dt
        \le \beta \|\bd{y} - \bd{x}\|_2.
    \]
    Thus, $f$ is $\beta$-Lipschitz.
\end{proof}

\begin{exercise}[First-order characterizations of convexity]
    Let $f : \bb{R}^n \to \bb{R}$ be differentiable.
    The following statements are equivalent:

    \begin{enumerate}
        \item $f$ is convex.

        \item For all $\bd{x}, \bd{y} \in \bb{R}^n$,
              \[
                  f(\bd{y})
                  \ge
                  f(\bd{x})
                  +
                  \langle \nabla f(\bd{x}), \bd{y} - \bd{x} \rangle.
              \]

        \item For all $\bd{x}, \bd{y} \in \bb{R}^n$,
              \[
                  \langle \nabla f(\bd{x}) - \nabla f(\bd{y}), \bd{x} - \bd{y} \rangle \ge 0.
              \]
    \end{enumerate}
\end{exercise}
\begin{proof}
    We prove the equivalence by showing $(1)\Rightarrow(2)\Rightarrow(3)\Rightarrow(1)$.

    \medskip

    \noindent
    $(1)\Rightarrow(2)$.
    Assume that $f$ is convex.
    Fix $\bd{x}, \bd{y} \in \bb{R}^n$ and define
    \[
        \phi(t) := f(\bd{x} + t(\bd{y} - \bd{x})), \quad t \in [0,1].
    \]
    Since $f$ is convex and differentiable, $\phi$ is convex and differentiable.
    By convexity of $\phi$,
    \[
        \phi(1) \ge \phi(0) + \phi'(0).
    \]
    Using the chain rule,
    \[
        \phi'(0)
        =
        \langle \nabla f(\bd{x}), \bd{y} - \bd{x} \rangle.
    \]
    Substituting $\phi(1)=f(\bd{y})$ and $\phi(0)=f(\bd{x})$ yields
    \[
        f(\bd{y})
        \ge
        f(\bd{x})
        +
        \langle \nabla f(\bd{x}), \bd{y} - \bd{x} \rangle.
    \]

    \medskip

    \noindent
    $(2)\Rightarrow(3)$.
    Assume that (2) holds.
    Applying (2) with $(\bd{x},\bd{y})$ gives
    \[
        f(\bd{y})
        \ge
        f(\bd{x})
        +
        \langle \nabla f(\bd{x}), \bd{y} - \bd{x} \rangle.
    \]
    Applying (2) with $(\bd{y},\bd{x})$ gives
    \[
        f(\bd{x})
        \ge
        f(\bd{y})
        +
        \langle \nabla f(\bd{y}), \bd{x} - \bd{y} \rangle.
    \]
    Adding the two inequalities and simplifying yields
    \[
        \langle \nabla f(\bd{x}) - \nabla f(\bd{y}),
        \bd{x} - \bd{y} \rangle \ge 0.
    \]

    \medskip

    \noindent
    $(3)\Rightarrow(1)$.
    Assume that (3) holds.
    Fix $\bd{x}, \bd{y} \in \bb{R}^n$ and define
    \[
        \bld{\gamma}(t) := (1-t)\bd{x} + t\bd{y},
        \quad t \in [0,1],
    \]
    and
    \[
        \phi(t) := f(\bld{\gamma}(t)).
    \]
    By the chain rule,
    \[
        \phi'(t)
        =
        \langle \nabla f(\bld{\gamma}(t)), \bd{y} - \bd{x} \rangle.
    \]
    Applying (3) with $\bd{x}=\bld{\gamma}(t)$ and $\bd{y}=\bd{x}$ gives
    \[
        \langle \nabla f(\bld{\gamma}(t)) - \nabla f(\bd{x}),
        \bld{\gamma}(t) - \bd{x} \rangle \ge 0.
    \]
    Since $\bld{\gamma}(t)-\bd{x}=t(\bd{y}-\bd{x})$, this implies
    \[
        \phi'(t) \ge \phi'(0).
    \]
    Thus $\phi'$ is nondecreasing on $[0,1]$, and hence $\phi$ is convex.
    Therefore,
    \[
        f(\bld{\gamma}(t))
        \le
        (1-t)f(\bd{x}) + t f(\bd{y}),
        \quad \text{for all } t \in [0,1].
    \]
    This is precisely the definition of convexity of $f$.
\end{proof}


\begin{remark}
    When $n=1$, condition (3) reduces to
    \[
        (f'(x) - f'(y))(x - y) \ge 0,
    \]
    which is equivalent to monotonicity of $f'$.
    Hence, a differentiable function $f : \bb{R} \to \bb{R}$ is convex if and only if
    its derivative is nondecreasing.
\end{remark}

\begin{lemma}[Smoothness Upper Bound or Descent Lemma]
    Let $f \in C^1_\beta(\bb{X})$, where $\bb{X} \subset \bb{R}^n$.
    Then, for all $\bd{x}, \bd{y} \in \bb{X}$,
    \[
        f(\bd{y})
        \le
        f(\bd{x})
        +
        \langle \nabla f(\bd{x}), \bd{y} - \bd{x} \rangle
        +
        \frac{\beta}{2}\|\bd{y} - \bd{x}\|_2^2.
    \]
\end{lemma}
\begin{proof}
    Fix $\bd{x}, \bd{y} \in \bb{X}$ and define
    \[
        \bld{\gamma}(t) := \bd{x} + t(\bd{y} - \bd{x}),
        \quad t \in [0,1],
    \]
    and
    \[
        \phi(t) := f(\bld{\gamma}(t)).
    \]
    By the chain rule,
    \[
        \phi'(t)
        =
        \langle \nabla f(\bld{\gamma}(t)), \bd{y} - \bd{x} \rangle.
    \]
    Using the fundamental theorem of calculus,
    \[
        f(\bd{y}) - f(\bd{x})
        =
        \phi(1) - \phi(0)
        =
        \int_0^1 \phi'(t)\, dt.
    \]
    Add and subtract $\nabla f(\bd{x})$ inside the integrand:
    \[
        \begin{aligned}
            f(\bd{y}) - f(\bd{x})
             & =
            \int_0^1
            \langle \nabla f(\bd{x}), \bd{y} - \bd{x} \rangle \, dt \\
             & \quad +
            \int_0^1
            \langle \nabla f(\bld{\gamma}(t)) - \nabla f(\bd{x}),
            \bd{y} - \bd{x} \rangle \, dt.
        \end{aligned}
    \]
    The first term gives
    \[
        \langle \nabla f(\bd{x}), \bd{y} - \bd{x} \rangle.
    \]
    For the second term, by Cauchy–Schwarz and $\beta$-Lipschitz continuity of $\nabla f$,
    \[
        \begin{aligned}
            \left|
            \langle \nabla f(\bld{\gamma}(t)) - \nabla f(\bd{x}),
            \bd{y} - \bd{x} \rangle
            \right|
             & \le
            \|\nabla f(\bld{\gamma}(t)) - \nabla f(\bd{x})\|_2
            \|\bd{y} - \bd{x}\|_2 \\
             & \le
            \beta t \|\bd{y} - \bd{x}\|_2^2.
        \end{aligned}
    \]
    Integrating over $t \in [0,1]$ yields
    \[
        \int_0^1 \beta t \|\bd{y} - \bd{x}\|_2^2 \, dt
        =
        \frac{\beta}{2}\|\bd{y} - \bd{x}\|_2^2.
    \]
    Combining all terms gives
    \[
        f(\bd{y})
        \le
        f(\bd{x})
        +
        \langle \nabla f(\bd{x}), \bd{y} - \bd{x} \rangle
        +
        \frac{\beta}{2}\|\bd{y} - \bd{x}\|_2^2.
    \]
\end{proof}

\begin{lemma}[Descent property of gradient descent]
    \label{lem:smoothness}
    Let $f \in C^1_\beta(\bb{X})$ and let $\gamma \in (0, 2/\beta]$.
    Consider the gradient descent iteration
    \[
        \bd{x}_{k+1} = \bd{x}_k - \gamma \nabla f(\bd{x}_k),
        \quad \bd{x}_0 \in \bb{X}.
    \]
    Then, for all $k \ge 0$,
    \[
        f(\bd{x}_{k+1}) \le f(\bd{x}_k),
    \]
    and more precisely,
    \[
        f(\bd{x}_{k+1})
        \le
        f(\bd{x}_k)
        -
        \gamma\Bigl(1 - \tfrac{\beta\gamma}{2}\Bigr)
        \|\nabla f(\bd{x}_k)\|_2^2.
    \]
\end{lemma}\begin{proof}
    Fix $k \ge 0$.
    Apply Lemma~\ref{lem:smoothness} (the smoothness upper bound) with
    \[
        \bd{x} = \bd{x}_k,
        \qquad
        \bd{y} = \bd{x}_{k+1}.
    \]
    Since
    \[
        \bd{x}_{k+1} - \bd{x}_k
        =
        - \gamma \nabla f(\bd{x}_k),
    \]
    Lemma~\ref{lem:smoothness} yields
    \[
        \begin{aligned}
            f(\bd{x}_{k+1})
             & \le
            f(\bd{x}_k)
            +
            \langle \nabla f(\bd{x}_k),
            \bd{x}_{k+1} - \bd{x}_k \rangle
            +
            \frac{\beta}{2}\|\bd{x}_{k+1} - \bd{x}_k\|_2^2 \\
             & =
            f(\bd{x}_k)
            -
            \gamma \|\nabla f(\bd{x}_k)\|_2^2
            +
            \frac{\beta}{2}\gamma^2
            \|\nabla f(\bd{x}_k)\|_2^2.
        \end{aligned}
    \]
    Rearranging gives
    \[
        f(\bd{x}_{k+1})
        \le
        f(\bd{x}_k)
        -
        \gamma\Bigl(1 - \tfrac{\beta\gamma}{2}\Bigr)
        \|\nabla f(\bd{x}_k)\|_2^2.
    \]
    If $\gamma \in (0,2/\beta]$, then
    $1 - \tfrac{\beta\gamma}{2} \ge 0$, and hence
    $f(\bd{x}_{k+1}) \le f(\bd{x}_k)$.
\end{proof}
\begin{theorem}
    Let $f \in C^1_\beta(\bb{R}^n)$ be convex and let
    $\bd{x}^\ast \in \arg\min f$ with $p^\ast := f(\bd{x}^\ast)$.
    Consider gradient descent with step size $\gamma \in (0, 1/\beta]$:
    \[
        \bd{x}_{k+1} = \bd{x}_k - \gamma \nabla f(\bd{x}_k).
    \]
    Define
    \[
        \delta_k := f(\bd{x}_k) - p^\ast.
    \]
    Then:
    \begin{enumerate}
        \item $\{f(\bd{x}_k)\}$ is nonincreasing and bounded below by $p^\ast$;
        \item $\sum_{k=0}^\infty \|\nabla f(\bd{x}_k)\|_2^2 < \infty$;
        \item there exists $C > 0$ such that
              \[
                  \delta_k \le \frac{C}{k}
                  \quad \text{for all } k \ge 1.
              \]
    \end{enumerate}
\end{theorem}

\begin{proof}
    By the descent lemma, for $\gamma \le 1/\beta$,
    \[
        f(\bd{x}_{k+1})
        \le
        f(\bd{x}_k)
        -
        \frac{\gamma}{2}\|\nabla f(\bd{x}_k)\|_2^2.
    \]
    Subtracting $p^\ast$ from both sides yields
    \[
        \delta_{k+1}
        \le
        \delta_k
        -
        \frac{\gamma}{2}\|\nabla f(\bd{x}_k)\|_2^2.
        \tag{1}
    \]
    Hence $\{\delta_k\}$ is nonincreasing and bounded below by $0$, and therefore
    \[
        \delta_k \to v \ge 0.
    \]

    Summing inequality \emph{(1)} from $k=0$ to $N-1$ gives
    \[
        \delta_N
        \le
        \delta_0
        -
        \frac{\gamma}{2}
        \sum_{k=0}^{N-1}
        \|\nabla f(\bd{x}_k)\|_2^2.
    \]
    Since $\delta_N \ge 0$, we obtain
    \[
        \sum_{k=0}^{\infty}
        \|\nabla f(\bd{x}_k)\|_2^2
        \le
        \frac{2}{\gamma}\delta_0
        < \infty.
        \tag{2}
    \]

    By convexity of $f$,
    \[
        f(\bd{x}_k) - p^\ast
        \le
        \langle \nabla f(\bd{x}_k), \bd{x}_k - \bd{x}^\ast \rangle.
    \]
    Using the Cauchy--Schwarz inequality,
    \[
        \delta_k
        \le
        \|\nabla f(\bd{x}_k)\|_2
        \|\bd{x}_k - \bd{x}^\ast\|_2.
    \]
    Since $\{f(\bd{x}_k)\}$ is decreasing and $f$ is coercive on level sets,
    there exists $R > 0$ such that
    \[
        \|\bd{x}_k - \bd{x}^\ast\|_2 \le R
        \quad \text{for all } k.
    \]

    The boundedness of $\{\bd{x}_k\}$ is established later using Fej\'er monotonicity;
    see Remark~\ref{rem:fejer-boundedness}.

    Thus,
    \[
        \delta_k
        \le
        R \|\nabla f(\bd{x}_k)\|_2.
        \tag{3}
    \]

    Combining \emph{(1)} and \emph{(3)}, we obtain
    \[
        \delta_{k+1}
        \le
        \delta_k
        -
        \frac{\gamma}{2R^2}
        \delta_k^2.
    \]
    This implies
    \[
        \frac{1}{\delta_{k+1}}
        \ge
        \frac{1}{\delta_k}
        +
        \frac{\gamma}{2R^2}.
    \]
    Iterating,
    \[
        \frac{1}{\delta_k}
        \ge
        \frac{1}{\delta_0}
        +
        k\frac{\gamma}{2R^2}.
    \]
    Taking reciprocals yields
    \[
        \delta_k
        \le
        \frac{2R^2}{\gamma k}
        = \frac{C}{k},
        \quad
        C := \frac{2R^2}{\gamma}.
    \]
\end{proof}
\begin{remark}\label{rem:fejer-boundedness}
    The boundedness assumption used in the proof above is not restrictive.
    In fact, for $\gamma \in (0,1/\beta]$, the gradient descent iterates
    $\{\bd{x}_k\}$ are Fej\'er monotone with respect to $\arg\min f$,
    which implies boundedness and convergence.
\end{remark}


\begin{lemma}[Baillon-Haddad]\label{lem:baillon-haddad}
    Let $f : \bb{R}^n \to \bb{R}$ be convex and continuously differentiable.
    If $\nabla f$ is $\beta$-Lipschitz, then $\nabla f$ is $\frac{1}{\beta}$-cocoercive, i.e.,
    for all $\bd{x}, \bd{y} \in \bb{R}^n$,
    \[
        \langle \nabla f(\bd{x}) - \nabla f(\bd{y}),
        \bd{x} - \bd{y} \rangle
        \ge
        \frac{1}{\beta}
        \|\nabla f(\bd{x}) - \nabla f(\bd{y})\|_2^2.
    \]
\end{lemma}

\begin{proof}
    Since $f \in C^1_\beta(\bb{R}^n)$, the descent lemma gives, for all
    $\bd{x}, \bd{y} \in \bb{R}^n$,
    \[
        f(\bd{y})
        \le
        f(\bd{x})
        +
        \langle \nabla f(\bd{x}), \bd{y} - \bd{x} \rangle
        +
        \frac{\beta}{2}\|\bd{y} - \bd{x}\|_2^2.
        \tag{1}
    \]
    Exchanging the roles of $\bd{x}$ and $\bd{y}$ yields
    \[
        f(\bd{x})
        \le
        f(\bd{y})
        +
        \langle \nabla f(\bd{y}), \bd{x} - \bd{y} \rangle
        +
        \frac{\beta}{2}\|\bd{x} - \bd{y}\|_2^2.
        \tag{2}
    \]
    Adding \emph{(1)} and \emph{(2)} gives
    \[
        \langle \nabla f(\bd{x}) - \nabla f(\bd{y}),
        \bd{x} - \bd{y} \rangle
        \le
        \beta \|\bd{x} - \bd{y}\|_2^2.
        \tag{3}
    \]

    On the other hand, since $\nabla f$ is $\beta$-Lipschitz,
    \[
        \|\nabla f(\bd{x}) - \nabla f(\bd{y})\|_2^2
        \le
        \beta
        \langle \nabla f(\bd{x}) - \nabla f(\bd{y}),
        \bd{x} - \bd{y} \rangle.
    \]
    Rearranging yields the desired inequality:
    \[
        \langle \nabla f(\bd{x}) - \nabla f(\bd{y}),
        \bd{x} - \bd{y} \rangle
        \ge
        \frac{1}{\beta}
        \|\nabla f(\bd{x}) - \nabla f(\bd{y})\|_2^2.
    \]
\end{proof}

\begin{definition}[Fej\'er monotonicity]
    Let $\mathcal{C} \subset \bb{R}^n$ be nonempty.
    A sequence $\{\bd{x}_k\} \subset \bb{R}^n$ is said to be
    \emph{Fej\'er monotone} with respect to $\mathcal{C}$ if
    \[
        \|\bd{x}_{k+1} - \bd{z}\|_2
        \le
        \|\bd{x}_k - \bd{z}\|_2
        \quad \text{for all } \bd{z} \in \mathcal{C} \text{ and all } k \ge 0.
    \]
\end{definition}

\begin{lemma}
    Let $f \in C^1_\beta(\bb{R}^n)$ be convex and let
    $\mathcal{X}^\ast := \arg\min f \neq \varnothing$.
    Assume $\gamma \in (0,1/\beta]$ and consider
    \[
        \bd{x}_{k+1} = \bd{x}_k - \gamma \nabla f(\bd{x}_k).
    \]
    Then $\{\bd{x}_k\}$ is Fej\'er monotone with respect to $\mathcal{X}^\ast$.
\end{lemma}
% \begin{proof}
%     Fix $\bd{x}^\ast \in \mathcal{X}^\ast$.
%     We compute
%     \[
%         \bd{x}_{k+1}-\bd{x}^\ast
%         =
%         \bd{x}_k-\bd{x}^\ast
%         -
%         \gamma \nabla f(\bd{x}_k).
%     \]
%     Taking squared norms,
%     \[
%         \begin{aligned}
%             \|\bd{x}_{k+1}-\bd{x}^\ast\|_2^2
%              & =
%             \|\bd{x}_k-\bd{x}^\ast\|_2^2
%             -2\gamma\langle\nabla f(\bd{x}_k),\bd{x}_k-\bd{x}^\ast\rangle
%             +\gamma^2\|\nabla f(\bd{x}_k)\|_2^2.
%         \end{aligned}
%     \]
%     By convexity,
%     \[
%         \langle\nabla f(\bd{x}_k),\bd{x}_k-\bd{x}^\ast\rangle
%         \ge f(\bd{x}_k)-f(\bd{x}^\ast).
%     \]
%     By $\beta$-smoothness,
%     \[
%         \|\nabla f(\bd{x}_k)\|_2^2
%         \le 2\beta\bigl(f(\bd{x}_k)-f(\bd{x}^\ast)\bigr).
%     \]
%     Substituting,
%     \[
%         \|\bd{x}_{k+1}-\bd{x}^\ast\|_2^2
%         \le
%         \|\bd{x}_k-\bd{x}^\ast\|_2^2
%         -2\gamma(1-\beta\gamma)\bigl(f(\bd{x}_k)-f(\bd{x}^\ast)\bigr).
%     \]
%     Since $\gamma\le 1/\beta$, the last term is nonnegative, hence
%     \[
%         \|\bd{x}_{k+1}-\bd{x}^\ast\|_2
%         \le
%         \|\bd{x}_k-\bd{x}^\ast\|_2.
%     \]
% \end{proof}
\begin{proof}
    Fix $\bd{x}^\ast \in \mathcal{X}^\ast$.
    We compute
    \[
        \bd{x}_{k+1}-\bd{x}^\ast
        =
        \bd{x}_k-\bd{x}^\ast
        -
        \gamma \nabla f(\bd{x}_k).
    \]
    Taking squared norms,
    \[
        \begin{aligned}
            \|\bd{x}_{k+1}-\bd{x}^\ast\|_2^2
             & =
            \|\bd{x}_k-\bd{x}^\ast\|_2^2
            -2\gamma\langle\nabla f(\bd{x}_k),\bd{x}_k-\bd{x}^\ast\rangle
            +\gamma^2\|\nabla f(\bd{x}_k)\|_2^2.
        \end{aligned}
    \]

    By convexity of $f$,
    \[
        \langle\nabla f(\bd{x}_k),\bd{x}_k-\bd{x}^\ast\rangle
        \ge
        f(\bd{x}_k)-f(\bd{x}^\ast).
        \tag{1}
    \]

    Since $f$ is convex and $\beta$-smooth, the Baillon--Haddad ~\ref{lem:baillon-haddad} implies that
    $\nabla f$ is $1/\beta$-cocoercive. In particular, using
    $\nabla f(\bd{x}^\ast)=0$, we obtain
    \[
        \|\nabla f(\bd{x}_k)\|_2^2
        \le
        \beta\,
        \langle\nabla f(\bd{x}_k),\bd{x}_k-\bd{x}^\ast\rangle.
        \tag{2}
    \]

    Substituting \emph{(1)} and \emph{(2)} into the norm identity yields
    \[
        \begin{aligned}
            \|\bd{x}_{k+1}-\bd{x}^\ast\|_2^2
             & \le
            \|\bd{x}_k-\bd{x}^\ast\|_2^2
            -2\gamma\langle\nabla f(\bd{x}_k),\bd{x}_k-\bd{x}^\ast\rangle
            +\beta\gamma^2\langle\nabla f(\bd{x}_k),\bd{x}_k-\bd{x}^\ast\rangle \\
             & =
            \|\bd{x}_k-\bd{x}^\ast\|_2^2
            -\gamma(2-\beta\gamma)
            \langle\nabla f(\bd{x}_k),\bd{x}_k-\bd{x}^\ast\rangle.
        \end{aligned}
    \]

    Since $\gamma \le 1/\beta$, the coefficient $(2-\beta\gamma)$ is nonnegative, and
    using \emph{(1)} once more, we conclude
    \[
        \|\bd{x}_{k+1}-\bd{x}^\ast\|_2^2
        \le
        \|\bd{x}_k-\bd{x}^\ast\|_2^2.
    \]
    Hence,
    \[
        \|\bd{x}_{k+1}-\bd{x}^\ast\|_2
        \le
        \|\bd{x}_k-\bd{x}^\ast\|_2,
    \]
    which proves Fej\'er monotonicity.
\end{proof}

\begin{lemma}
    Let $\{\bd{x}_k\}$ be Fej\'er monotone with respect to a closed set
    $\mathcal{C}\subset\bb{R}^n$.
    If there exists a subsequence $\bd{x}_{k_j}\to\bd{x}^\ast\in\mathcal{C}$,
    then $\bd{x}_k\to\bd{x}^\ast$.
\end{lemma}

\begin{proof}
    Since $\{\bd{x}_k\}$ is Fej\'er monotone with respect to $\mathcal{C}$,
    it is bounded.
    Hence, by the Bolzano-Weierstrass theorem, it admits a convergent subsequence
    $\bd{x}_{k_j}\to\bd{x}^\ast\in\mathcal{C}$.

    Fix $\bd{z}=\bd{x}^\ast$.
    By Fej\'er monotonicity,
    \[
        \|\bd{x}_{k+1}-\bd{x}^\ast\|_2
        \le
        \|\bd{x}_k-\bd{x}^\ast\|_2,
    \]
    so the sequence $\|\bd{x}_k-\bd{x}^\ast\|_2$ is nonincreasing and
    converges to some $\ell\ge0$.

    Along the subsequence,
    \[
        \|\bd{x}_{k_j}-\bd{x}^\ast\|_2 \to 0,
    \]
    hence $\ell=0$.
    Therefore,
    \[
        \bd{x}_k \to \bd{x}^\ast.
    \]
\end{proof}

\begin{lemma}[Fejér monotonicity of gradient descent iterates]
    Let $f \in C^1_\beta(\bb{R}^n)$ be convex and let
    $\bd{x}^\ast \in \arg\min f$.
    Assume $\gamma \in (0, 1/\beta]$ and consider the gradient descent iteration
    \[
        \bd{x}_{k+1} = \bd{x}_k - \gamma \nabla f(\bd{x}_k).
    \]
    Then the sequence $\{\bd{x}_k\}$ is Fejér monotone with respect to the solution set
    $\mathcal{X}^\ast := \arg\min f$, i.e.,
    \[
        \|\bd{x}_{k+1} - \bd{x}^\ast\|_2
        \le
        \|\bd{x}_k - \bd{x}^\ast\|_2
        \quad \text{for all } \bd{x}^\ast \in \mathcal{X}^\ast,
    \]
    and in particular $\{\bd{x}_k\}$ is bounded.
\end{lemma}
\begin{proof}
    Fix $\bd{x}^\ast \in \mathcal{X}^\ast$.
    Using the update rule,
    \[
        \bd{x}_{k+1} - \bd{x}^\ast
        =
        \bd{x}_k - \bd{x}^\ast
        -
        \gamma \nabla f(\bd{x}_k).
    \]
    Taking squared norms,
    \[
        \begin{aligned}
            \|\bd{x}_{k+1} - \bd{x}^\ast\|_2^2
             & =
            \|\bd{x}_k - \bd{x}^\ast\|_2^2
            - 2\gamma
            \langle \nabla f(\bd{x}_k), \bd{x}_k - \bd{x}^\ast \rangle
            + \gamma^2 \|\nabla f(\bd{x}_k)\|_2^2.
        \end{aligned}
    \]
    By convexity of $f$ and optimality of $\bd{x}^\ast$,
    \[
        \langle \nabla f(\bd{x}_k), \bd{x}_k - \bd{x}^\ast \rangle
        \ge
        f(\bd{x}_k) - f(\bd{x}^\ast).
        \tag{1}
    \]
    By $\beta$-smoothness,
    \[
        \|\nabla f(\bd{x}_k)\|_2^2
        \le
        2\beta \bigl(f(\bd{x}_k) - f(\bd{x}^\ast)\bigr).
        \tag{2}
    \]
    Substituting \emph{(1)} and \emph{(2)} into the squared norm expression yields
    \[
        \begin{aligned}
            \|\bd{x}_{k+1} - \bd{x}^\ast\|_2^2
             & \le
            \|\bd{x}_k - \bd{x}^\ast\|_2^2
            - 2\gamma \bigl(f(\bd{x}_k) - f(\bd{x}^\ast)\bigr)
            + 2\beta\gamma^2 \bigl(f(\bd{x}_k) - f(\bd{x}^\ast)\bigr) \\
             & =
            \|\bd{x}_k - \bd{x}^\ast\|_2^2
            - 2\gamma(1 - \beta\gamma)
            \bigl(f(\bd{x}_k) - f(\bd{x}^\ast)\bigr).
        \end{aligned}
    \]
    Since $\gamma \le 1/\beta$, the coefficient
    $2\gamma(1-\beta\gamma)$ is nonnegative, and hence
    \[
        \|\bd{x}_{k+1} - \bd{x}^\ast\|_2^2
        \le
        \|\bd{x}_k - \bd{x}^\ast\|_2^2.
    \]
    Therefore, $\{\bd{x}_k\}$ is Fejér monotone with respect to $\mathcal{X}^\ast$.
    In particular, the sequence $\{\bd{x}_k\}$ is bounded.
\end{proof}
\begin{lemma}[Convergence of Fej\'er monotone sequences]
    \label{lem:convergence_fejer_monotone_limit}
    Let $\mathcal{C} \subset \bb{R}^n$ be nonempty and closed, and let
    $\{\bd{x}_k\} \subset \bb{R}^n$ be Fej\'er monotone with respect to
    $\mathcal{C}$, i.e.,
    \[
        \|\bd{x}_{k+1} - \bd{z}\|_2
        \le
        \|\bd{x}_k - \bd{z}\|_2
        \quad \text{for all } \bd{z} \in \mathcal{C},\ \forall k \ge 0.
    \]
    If there exists a subsequence $\{\bd{x}_{k_j}\}$ and a point
    $\bd{x}^\ast \in \mathcal{C}$ such that
    \[
        \bd{x}_{k_j} \to \bd{x}^\ast,
    \]
    then the whole sequence converges:
    \[
        \bd{x}_k \to \bd{x}^\ast.
    \]
\end{lemma}
\begin{proof}
    Fix $\bd{z} = \bd{x}^\ast \in \mathcal{C}$.
    By Fej\'er monotonicity, the sequence
    \[
        d_k := \|\bd{x}_k - \bd{x}^\ast\|_2
    \]
    is nonincreasing and bounded below by $0$.
    Hence, there exists $\ell \ge 0$ such that
    \[
        d_k \to \ell.
    \]

    On the other hand, along the convergent subsequence,
    \[
        d_{k_j}
        =
        \|\bd{x}_{k_j} - \bd{x}^\ast\|_2
        \to 0.
    \]
    Therefore, $\ell = 0$.
    It follows that
    \[
        \|\bd{x}_k - \bd{x}^\ast\|_2 \to 0,
    \]
    and hence $\bd{x}_k \to \bd{x}^\ast$.
\end{proof}

\begin{lemma}[Convergence via boundedness and Fej\'er monotonicity]

    Let $f \in C^1_\beta(\bb{R}^n)$ be convex and let
    $\mathcal{X}^\ast := \arg\min f \neq \varnothing$.
    Assume $\gamma \in (0,1/\beta]$ and consider the gradient descent iteration
    \[
        \bd{x}_{k+1} = \bd{x}_k - \gamma \nabla f(\bd{x}_k).
    \]
    Then the sequence $\{\bd{x}_k\}$ converges to a point
    $\bd{x}^\ast \in \mathcal{X}^\ast$.
\end{lemma}
\begin{proof}
    By Fej\'er monotonicity with respect to $\mathcal{X}^\ast$, the sequence
    $\{\bd{x}_k\}$ is bounded.
    Hence, by the Bolzano--Weierstrass theorem, there exists a subsequence
    $\{\bd{x}_{k_j}\}$ and a point $\bd{x}^\ast \in \bb{R}^n$ such that
    \[
        \bd{x}_{k_j} \to \bd{x}^\ast.
    \]

    By the descent lemma,
    \[
        f(\bd{x}_{k+1})
        \le
        f(\bd{x}_k)
        -
        \frac{\gamma}{2}\|\nabla f(\bd{x}_k)\|_2^2,
    \]
    which implies
    \[
        \|\nabla f(\bd{x}_k)\|_2 \to 0.
    \]
    By continuity of $\nabla f$,
    \[
        \nabla f(\bd{x}^\ast) = 0.
    \]

    Since $f$ is convex, this implies $\bd{x}^\ast \in \mathcal{X}^\ast$.

    By Fej\'er monotonicity and  Lemma~\ref{lem:convergence_fejer_monotone_limit}, the sequence $\{\bd{x}_k\}$ converges
    to $\bd{x}^\ast \in \mathcal{X}^\ast$.

\end{proof}
\begin{lemma}
    Let $f : \bb{R}^n \to \bb{R}$ be differentiable.
    The following statements are equivalent:
    \begin{enumerate}
        \item $f$ is convex.
        \item For all $\bd{x}, \bd{y} \in \bb{R}^n$,
              \[
                  f(\bd{y})
                  \ge
                  f(\bd{x})
                  +
                  \langle \nabla f(\bd{x}), \bd{y} - \bd{x} \rangle.
              \]
        \item For all $\bd{x}, \bd{y} \in \bb{R}^n$,
              \[
                  \langle \nabla f(\bd{x}) - \nabla f(\bd{y}),
                  \bd{x} - \bd{y} \rangle \ge 0.
              \]
    \end{enumerate}
\end{lemma}