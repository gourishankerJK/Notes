\section{Introduction }

\begin{definition}[Optimization Problem and Optimal Value]
    Consider the optimization problem
    \[
        \min_{x \in X} f(x),
    \]
    where \( f : X \to \bb{R} \).

    The associated \emph{optimal value} is defined as
    \[
        p^\star := \inf_{x \in X} f(x).
    \]

    The infimum is used instead of the minimum since, in general, the function \( f \) may fail to attain its minimum over the set \( X \). That is, there may exist no point \( x^\star \in X \) such that
    \[
        f(x^\star) = \min_{x \in X} f(x).
    \]

    Nevertheless, the set \( \{ f(x) : x \in X \} \subset \bb{R} \) always admits a greatest lower bound in
    \( \bb{R} \cup \{-\infty\} \), which is precisely the infimum.

    If there exists a point \( x^\star \in X \) satisfying
    \[
        f(x^\star) = \inf_{x \in X} f(x),
    \]
    then the infimum is said to be \emph{attained}, and we may equivalently write
    \[
        p^\star = \min_{x \in X} f(x).
    \]
    In this case, \( x^\star \) is called an \emph{optimal solution}.
\end{definition}
\begin{definition}[Constrained optimization problem]
    Consider the constrained optimization problem
    \[
        \begin{aligned}
            \min_{x \in \bb{R}^n} \quad & f(x)      \\
            \text{subject to} \quad     & g(x) = 0,
        \end{aligned}
    \]
    where \( f : \bb{R}^n \to \bb{R} \) and \( g : \bb{R}^n \to \bb{R}^m \).

    The associated \emph{feasible set} is defined by
    \[
        X := \{ x \in \bb{R}^n : g(x) = 0 \}.
    \]
\end{definition}

\begin{definition}[Local minimum (constrained)]
    A point \( x^\star \in X \) is called a \emph{local minimum} of the constrained optimization problem if there exists \( \delta > 0 \) such that
    \[
        f(x^\star) \le f(x)
        \quad \text{for all } x \in X \cap B(x^\star,\delta),
    \]
    where
    \[
        B(x^\star,\delta)
        := \{ x \in \bb{R}^n : \|x - x^\star\| < \delta \}
    \]
    denotes the open ball of radius \( \delta \) centered at \( x^\star \).

    Equivalently, \( x^\star \) is a local minimum if no feasible point sufficiently close to \( x^\star \) achieves a strictly smaller objective value.
\end{definition}
\begin{center}
    \begin{tabular}{c|c|c}
         & \textbf{Smooth}                  & \textbf{Non-smooth} \\
        \hline
        \textbf{Convex}
         & \begin{tabular}{c}
               Convex Optimization \\
               \smiley
           \end{tabular}
         & \begin{tabular}{c}
               Advanced Convex Optimization \\
               (Proximal methods, subgradients)
           \end{tabular}
        \\
        \hline
        \textbf{Non-convex}
         & \begin{tabular}{c}
               Deep Learning
           \end{tabular}
         & \begin{tabular}{c}
               :(
           \end{tabular}
    \end{tabular}
\end{center}
\begin{marginfigure}
    \centering
    \begin{tikzpicture}[scale=0.8]
        \begin{scope}
            \draw[->] (-1.5,0) -- (1.5,0) node[right] {\footnotesize $x$};
            \draw[->] (0,-0.5) -- (0,2) node[above] {\footnotesize $|x|$};
            \draw[thick, blue] (-1.2,1.2) -- (0,0) -- (1.2,1.2);
        \end{scope}
        \begin{scope}[yshift=-3.5cm]
            \draw[->] (-1.5,0) -- (1.5,0) node[right] {\footnotesize $x$};
            \draw[->] (0,-1.5) -- (0,1.5) node[above] {\footnotesize $\partial |x|$};
            \draw[thick, red] (-1.5,-1) -- (0,-1);
            \draw[thick, red] (0,1) -- (1.5,1);
            \draw[thick, red] (0,-1) -- (0,1);
            \fill[red] (0,-1) circle (1.5pt);
            \fill[red] (0,1) circle (1.5pt);
            \node[below] at (-0.8,-1) {\scriptsize $-1$};
            \node[above] at (0.8,1) {\scriptsize $1$};
        \end{scope}
    \end{tikzpicture}
    \caption{The function $|x|$ and its subdifferential.}
    \label{fig:abs_subdiff}
\end{marginfigure}
\begin{example}[Soft-thresholding via subdifferential calculus]
    Consider the optimization problem
    \[
        \min_{x \in \bb{R}}
        \left\{
        \frac{1}{2}(x-a)^2 + \tau |x|
        \right\},
        \qquad \tau > 0.
    \]

    The objective function is convex but non-smooth due to the absolute value term.
    The subdifferential of \( |x| \) is the set-valued mapping
    \[
        \partial |x|
        =
        \begin{cases}
            \{1\},    & x > 0, \\[6pt]
            [-1,\,1], & x = 0, \\[6pt]
            \{-1\},   & x < 0.
        \end{cases}
    \]

    Define
    \[
        \phi(x) := \frac{1}{2}(x-a)^2 + \tau |x|.
    \]
    Since \( \phi \) is convex, a point \( x^\star \in \bb{R} \) is optimal if and only if
    \[
        0 \in \partial \phi(x^\star)
        = (x^\star - a) + \tau \partial |x^\star|.
    \]
    Equivalently,
    \[
        a - x^\star \in \tau \partial |x^\star|.
    \]

    A case-by-case analysis yields
    \[
        x^\star
        =
        \begin{cases}
            a - \tau, & a > \tau,     \\[6pt]
            0,        & |a| \le \tau, \\[6pt]
            a + \tau, & a < -\tau.
        \end{cases}
    \]

    This solution can be written compactly as the \emph{soft-thresholding operator}
    \[
        x^\star
        = \mathrm{sign}(a)\,\max\{ |a| - \tau,\, 0 \}.
    \]
\end{example}
\section{Inclusion Problems, Convexity, and Smoothness}

\begin{definition}[Inclusion problem]
    Let \( X \) be a real Euclidean space.
    An \emph{inclusion problem} consists in finding
    \[
        x \in X \quad \text{such that} \quad 0 \in T(x),
    \]
    where \( T : X \rightrightarrows X \) is a set-valued operator.
\end{definition}

\begin{remark}
    Many convex optimization problems can be written as inclusion problems by taking
    \( T = \partial f \), where \( \partial f \) denotes the subdifferential of a convex function.
\end{remark}

\begin{definition}[Convex optimization problem]
    Let \( C \subset X \) be a convex set and let
    \( f : X \to \bb{R} \) be a convex function.
    The problem
    \[
        \min_{x \in C} f(x)
    \]
    is called a \emph{convex optimization problem}.
\end{definition}

\begin{proposition}
    If both the objective function \( f \) and the constraint set \( C \) are convex,
    then every local minimizer is a global minimizer.
\end{proposition}


\begin{proposition}
    Let \( f : \bb{R}^n \to \bb{R} \) be twice continuously differentiable.
    If
    \[
        \nabla^2 f(x) \succeq 0
        \quad \text{for all } x \in \bb{R}^n,
    \]
    then \( f \) is convex.
    If
    \[
        \nabla^2 f(x) \succ 0
        \quad \text{for all } x \in \bb{R}^n,
    \]
    then \( f \) is strictly convex.
\end{proposition}

\subsection{Euclidean spaces and induced norms}

\begin{definition}[Euclidean space]
    A real Euclidean space \( X \) is a real vector space equipped with an inner product
    \[
        \iprod{x}{y}.
    \]
    The induced norm is defined by
    \[
        \|x\| := \sqrt{\iprod{x}{x}}.
    \]
\end{definition}

\begin{remark}
    The \( \ell^2 \)-norm on \( \bb{R}^n \) is induced by the standard inner product.
    The \( \ell^1 \)-norm is \emph{not} induced by any inner product.
\end{remark}


\begin{definition}
    The norm induces a metric
    \[
        d(x,y) := \|x-y\|,
    \]
    which defines a topology on \( X \).
\end{definition}

\begin{definition}[Convergence]
    A sequence \( (x_k) \subset X \) converges to \( x \in X \) if
    \[
        \|x_k - x\| \to 0.
    \]
\end{definition}

\begin{definition}[Continuity]
    A function \( f : X \to \bb{R} \) is continuous at \( x \in X \) if
    \[
        x_k \to x \quad \Rightarrow \quad f(x_k) \to f(x).
    \]
\end{definition}


\begin{definition}[Topological space]
    Let \( X \) be a set and let \( \tau \) be a collection of subsets of \( X \).
    The pair \( (X,\tau) \) is called a \emph{topological space} if
    \begin{itemize}
        \item \( \varnothing \in \tau \) and \( X \in \tau \),
        \item the union of any collection of sets in \( \tau \) belongs to \( \tau \),
        \item the intersection of any finite collection of sets in \( \tau \) belongs to \( \tau \).
    \end{itemize}
    The elements of \( \tau \) are called \emph{open sets}.
\end{definition}

\begin{definition}[Convergence in a topological space]
    Let \( (X,\tau) \) be a topological space and let \( (x_k) \subset X \).
    We say that \( x_k \) converges to \( x \in X \) if
    for every open set \( U \in \tau \) such that \( x \in U \),
    there exists \( N \in \bb{N} \) for which
    \[
        x_k \in U
        \quad \text{for all } k \ge N.
    \]
\end{definition}

\begin{remark}
    This definition of convergence depends only on the topology \( \tau \)
    and does not require any notion of distance or norm.
    In particular, convergence can be defined in spaces that are not metric spaces.
\end{remark}

\begin{proposition}
    Every convex function
    \( f : \bb{R}^n \to \bb{R} \)
    is continuous.
\end{proposition}


\begin{proposition}
    For \( 0 < p < 1 \), the set
    \[
        B_p := \{ x \in \bb{R}^n : \|x\|_p \le 1 \}
    \]
    is not convex.
\end{proposition}


\begin{definition}[Differentiability]
    A function \( f : \bb{R}^n \to \bb{R} \) is said to be differentiable at
    \( x \in \bb{R}^n \) if there exists \( z \in \bb{R}^n \) such that
    \[
        f(x+h)
        =
        f(x)
        + \iprod{z}{h}
        + o(\|h\|)
        \quad \text{as } h \to 0.
    \]
    In this case, \( z = \nabla f(x) \).
    where     The term \( o(\|h\|) \) satisfies
    \[
        \lim_{\|h\|\to 0}
        \frac{o(\|h\|)}{\|h\|} = 0.
    \]
\end{definition}
\begin{exercise}[Differentiability and inner products]
    Let \( f : \bb{R}^n \to \bb{R} \) be differentiable at \( x \in \bb{R}^n \) in the sense that
    there exists \( z \in \bb{R}^n \) such that
    \[
        f(x+h)
        =
        f(x)
        + \iprod{z}{h}
        + o(\|h\|)
        \quad \text{as } h \to 0,
    \]
    where
    \[
        \lim_{\|h\|\to 0} \frac{o(\|h\|)}{\|h\|} = 0.
    \]

    \begin{enumerate}
        \item Show that for a fixed inner product \( \iprod{\cdot}{\cdot} \), the vector \( z \) is unique.
        \item Show that the vector \( z \) depends on the choice of inner product.
        \item Show that the notion of differentiability does not depend on the choice of inner product on \( \bb{R}^n \).
        \item Show that all norms on \( \bb{R}^n \) are equivalent.
        \item Prove the Riesz representation theorem: every linear functional on a Euclidean space can be represented as an inner product.
    \end{enumerate}
\end{exercise}
\begin{proof}[Solution]
    \begin{enumerate}
        \item \emph{Uniqueness:}
              Assume that there exist \( z_1, z_2 \in \bb{R}^n \) such that
              \[
                  f(x+h) = f(x) + \iprod{z_1}{h} + o(\|h\|)
              \]
              and
              \[
                  f(x+h) = f(x) + \iprod{z_2}{h} + o(\|h\|).
              \]
              Subtracting the two expressions gives
              \[
                  \iprod{z_1 - z_2}{h} = o(\|h\|).
              \]
              Dividing by \( \|h\| \) and letting \( h \to 0 \),
              \[
                  \frac{\iprod{z_1 - z_2}{h}}{\|h\|} \to 0.
              \]
              Choosing \( h = z_1 - z_2 \) yields
              \[
                  \|z_1 - z_2\|^2 = 0,
              \]
              hence \( z_1 = z_2 \).

        \item \emph{Dependence on inner product:}
              Let \( f(x) = a^\TT x \), where \( a \in \bb{R}^n \).
              For the standard inner product,
              \[
                  f(x+h) = f(x) + \iprod{a}{h},
              \]
              so \( z = a \).

              Now consider a different inner product
              \[
                  \iprod{x}{y}_M := \iprod{Mx}{y},
              \]
              where \( M \) is symmetric positive definite.
              Then
              \[
                  f(x+h) = f(x) + \iprod{M^{-1}a}{h}_M.
              \]
              Hence the representing vector \( z \) changes with the inner product.

        \item \emph{Independence of differentiability:}
              Differentiability is equivalent to the existence of a linear map
              \( L : \bb{R}^n \to \bb{R} \) such that
              \[
                  f(x+h) = f(x) + L(h) + o(\|h\|).
              \]
              This definition does not involve any inner product.

              By the Riesz representation theorem, for any chosen inner product,
              there exists a unique vector \( z \) such that
              \[
                  L(h) = \iprod{z}{h}.
              \]
              Thus, while the vector \( z \) depends on the inner product,
              the existence of the derivative does not.

        \item \emph{Norm equivalence:}
              Let \( \|\cdot\|_a \) and \( \|\cdot\|_b \) be two norms on \( \bb{R}^n \).
              To show they are equivalent, we need to find \( c, C > 0 \) such that
              \( c \|x\|_a \le \|x\|_b \le C \|x\|_a \) for all \( x \).

              The unit sphere \( S = \{x : \|x\|_a = 1\} \) is compact.
              The function \( x \mapsto \|x\|_b \) is continuous on \( S \),
              so it attains a minimum \( c > 0 \) and a maximum \( C > 0 \).
              By homogeneity of norms, the result follows for all \( x \).

        \item \emph{Riesz representation theorem:}
              Let \( \{e_1,\dots,e_n\} \) be an orthonormal basis of \( X \).
              Define
              \[
                  z := \sum_{i=1}^n L(e_i)\, e_i.
              \]
              Then for any \( h = \sum_i h_i e_i \),
              \[
                  \iprod{z}{h}
                  =
                  \sum_{i=1}^n L(e_i) h_i
                  =
                  L(h).
              \]
              Uniqueness follows from non-degeneracy of the inner product.
    \end{enumerate}
\end{proof}

\begin{theorem}
    Let \( X \subset \bb{R}^n \) be a convex set and let
    \( f : X \to \bb{R} \) be convex and differentiable.
    If \( x^\star \in X \) satisfies
    \[
        \nabla f(x^\star) = 0,
    \]
    then \( x^\star \) is a global minimizer of \( f \) on \( X \).
\end{theorem}

\begin{proof}
    Since \( f \) is convex, for any \( x, y \in X \) and any \( t \in [0,1] \),
    \[
        f(tx + (1-t)y)
        \le
        t f(x) + (1-t) f(y).
    \]

    Fix \( x \in X \) and define the function
    \[
        \varphi(t) := f(x^\star + t(x - x^\star)),
        \qquad t \in [0,1].
    \]
    By convexity of \( f \), the function \( \varphi \) is convex on \( [0,1] \).

    Since \( f \) is differentiable, \( \varphi \) is differentiable and
    \[
        \varphi'(0)
        =
        \iprod{\nabla f(x^\star)}{x - x^\star}.
    \]

    By assumption, \( \nabla f(x^\star) = 0 \), hence
    \[
        \varphi'(0) = 0.
    \]

    For a convex function on an interval, the derivative is monotone non-decreasing.
    Therefore, for all \( t \in [0,1] \),
    \[
        \varphi(t) \ge \varphi(0).
    \]

    In particular, at \( t = 1 \),
    \[
        f(x) = \varphi(1) \ge \varphi(0) = f(x^\star).
    \]

    Since \( x \in X \) was arbitrary, \( x^\star \) is a global minimizer of \( f \).
\end{proof}
\subsection{Unconstrained and Constrained Optimization}

An \emph{unconstrained optimization problem} is of the form
\[
    \min_{x \in X} f(x),
\]
where typically \( X = \bb{R}^n \) and
\( f : \bb{R}^n \to \bb{R} \).

In this case, every point in \( X \) is feasible, and optimality is characterized
by stationarity conditions such as
\[
    \nabla f(x^\star) = 0.
\]

A \emph{constrained optimization problem} is given by
\[
    \min_{x \in C} f(x),
\]
where \( C \subset \bb{R}^n \) is a constraint set and
\( f : \bb{R}^n \to \bb{R} \).

\begin{definition}[Feasible set]
    The set
    \[
        C := \{ x \in \bb{R}^n : \text{all constraints are satisfied} \}
    \]
    is called the \emph{feasible set}.
    Points in \( C \) are called \emph{feasible points}.
\end{definition}



\begin{definition}[Indicator function]
    Let \( C \subset \bb{R}^n \).
    The indicator function \( \delta_C : \bb{R}^n \to \bb{R} \cup \{+\infty\} \) is defined as
    \[
        \delta_C(x)
        :=
        \begin{cases}
            0,       & x \in C,    \\[4pt]
            +\infty, & x \notin C.
        \end{cases}
    \]
\end{definition}

Then the constrained problem
\[
    \min_{x \in C} f(x)
\]
is equivalent to the unconstrained problem
\[
    \min_{x \in \bb{R}^n} \bigl( f(x) + \delta_C(x) \bigr).
\]




\begin{definition}[Lagrangian]
    Consider the constrained problem
    \[
        \begin{aligned}
            \min_{x \in \bb{R}^n} \quad & f(x)        \\
            \text{subject to} \quad     & g(x) \le 0,
        \end{aligned}
    \]
    where \( g : \bb{R}^n \to \bb{R}^m \).
    The \emph{Lagrangian} associated with the problem is
    \[
        \mathcal{L}(x,\lambda)
        :=
        f(x) + \iprod{\lambda}{g(x)},
        \qquad \lambda \in \bb{R}^m_+.
    \]
\end{definition}


\begin{proposition}[Optimality and saddle-point property]
    If \( x^\star \) is a solution of the constrained problem and
    \( \lambda^\star \ge 0 \) is a corresponding Lagrange multiplier, then
    \[
        \mathcal{L}(x^\star,\lambda)
        \le
        \mathcal{L}(x^\star,\lambda^\star)
        \le
        \mathcal{L}(x,\lambda^\star)
    \]
    for all feasible \( x \) and all \( \lambda \ge 0 \).
\end{proposition}

Thus, optimal solutions correspond to saddle points of the Lagrangian.


\begin{remark}[Constraint qualification]
    Interior-point conditions (such as Slater’s condition) ensure that
    Lagrange multipliers exist and that optimality conditions are valid.
\end{remark}

\begin{example}
    Consider the problem
    \[
        \min_{x \in \bb{R}} x
        \quad \text{subject to} \quad x \ge 0.
    \]
    The optimal solution is \( x^\star = 0 \), which lies on the boundary of the feasible set.

    If we perturb the constraint slightly, the gradient of the objective
    cannot be balanced by any Lagrange multiplier unless the feasible set
    has nonempty interior.
    This failure prevents the derivation of meaningful optimality conditions.
\end{example}

\begin{remark}
    Interior feasibility guarantees that constraints are not degenerate and
    ensures the validity of duality and KKT conditions.
    Without interior points, multipliers may fail to exist or be unbounded.
\end{remark}
